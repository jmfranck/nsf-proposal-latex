\begin{pappg}
    \textbf{Cover page information entered
    on fastlane, but organize here}
\end{pappg}
\begin{description}
    \item[Organization Name and Address:] \quad
    \item[Program
            Announcement/Solicitation/Program
        Description Number:]  \quad
        \begin{pappg}
            If the proposal is not submitted
            in response to a specific
            funding opportunity,
            proposers should select
            "Proposal and Award Policies and
            Procedures Guide." Proposers are
            advised to
            select "No Closing Date" when
            the proposal is not submitted in
            response to any relevant NSF
            funding
            opportunity.
        \end{pappg}
    \item[NSF Unit of Consideration:]  \quad
        \begin{pappg}
            Proposers must follow
            instructions for selection of an
            applicable NSF Division/Office
            and Program(s) to
            which the proposal should be
            directed.
        \end{pappg}
    \item[Title of Proposed Project]  \quad
        \begin{pappg}
            The title of the project must be
            brief, scientifically or
            technically valid, and suitable
            for use in the public
            press. NSF may edit the title of
            a project prior to making an
            award.
        \end{pappg}
    \item[Budget and Duration
        Information:]  \quad
        \begin{pappg}
            The proposed duration for which support is requested
            should be consistent with the nature and complexity
            of the proposed activity. The Foundation encourages
            proposers to request funding for durations of three
            to five years when such durations are necessary for
            completion of the proposed work and are technically
            and managerially advantageous. The requested start
            date should allow at least six months for NSF review,
            processing and decision. The PI should consult
            his/her organization’s SPO for unusual situations
            (e.g., a long lead time for procurement) that create
            problems regarding the proposed start date.
            Specification of a desired start date for the project
            is important and helpful to NSF staff; however,
            requests for specific start dates may not be met.
        \end{pappg}
    \item[Announcement and Consideration Information]  \quad
        \begin{pappg}
            This information is prefilled based on previously entered
            information.
        \end{pappg}
    \item[PI Information and co-PI Information]  \quad
        \begin{pappg}
            (Only need to enter co-PIs)

            Each individual's name and either NSF ID or primary
            registered e-mail address, must be entered in the
            boxes provided.
        \end{pappg}
    \item[Previous NSF Award]  \quad
    \item[Consideration by Other Federal Agencies]  \quad
    \item[Awardee Organization Information]  \quad
        \begin{pappg}
            The awardee organization name, address, NSF organization
            code, DUNS number and Employer
            Identification Number/Taxpayer Identification Number are
            derived from the profile information provided by
            the organization or pulled by NSF from the SAM database
            and are not entered when preparing the Cover
            Sheet.
        \end{pappg}
    \item[Primary Place of Performance] pre-filled
    \item[Other Information] checking boxes for Human subjects,
        etc.
\end{description}
