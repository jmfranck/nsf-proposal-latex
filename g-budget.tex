\section*{Budget Justification}

\begin{pappg}
    Each proposal must contain a budget for each year of support requested.
    The budget justification must be no more than five pages per proposal.
    The amounts for each budget line item requested must be documented and
    justified in the budget justification as specified below. For proposals
    that contain a subaward(s), \textbf{each subaward} must include a
    separate budget justification of no more than five pages. See Chapter
    II.C.2.g.(vi)(e) for further instructions on proposals that contain
    subawards.

    The proposal may request funds under any of the categories listed so
    long as the item and amount are considered necessary, reasonable,
    allocable, and allowable under 2 CFR § 200, Subpart E, NSF policy,
    and/or the program solicitation. For-profit entities are subject to the
    cost principles contained in the Federal Acquisition Regulation, Part
    31. Amounts and expenses budgeted also must be consistent with the
    proposing organization's policies and procedures and cost accounting
    practices used in accumulating and reporting costs.

    Proposals for major facilities also should consult NSF's Large
    Facilities Manual for additional budgetary preparation guidelines.
\end{pappg}

\subsection*{A. Senior Personnel}

\noindent{\bf A1.} Person One, Title and Affiliation, will serve as PI on this project and oversee blah blah. We request 1 month of summer salary per year for Dr. Person One, which is based on a rate of \$10,000/month and includes a 2\% annual escalation:
\newline
\begin{table}[h]
\begin{tabular}[c]{cccc}
Year 1: \$10,000 & Year 2: \$10,200 & Year 3: \$10,404 & Total: \textbf{\$30,604}
\end{tabular}
\end{table}
\newline
\newline
\noindent{\bf A2.} Person Two, Title and Affiliation, will serve as Co-PI on this project and will do nothing. We request 1 month of summer salary per year for Dr. Person Two, which is based on a rate of \$10,000/month and includes a 2\% annual escalation:
\newline
\begin{table}[h]
\begin{tabular}[c]{cccc}
Year 1: \$10,000 & Year 2: \$10,200 & Year 3: \$10,404 & Total: \textbf{\$30,604}
\end{tabular}
\end{table}

\subsection*{B. Other Personnel}
\noindent{\bf B3.} We request salary support for one PhD student, which is based on a rate of \$25,000/year and a 2\% annual escalation. The student will bring Persons One and Two coffee upon request.
\newline
\begin{table}[h]
\begin{tabular}[c]{cccc}
Year 1: \$25,000 & Year 2: \$25,500 & Year 3: \$26,010 & Total: \textbf{\$76,510}
\end{tabular}
\end{table}

\noindent{\bf B4.} We request salary support for two part-time undergraduate students, which is based on a rate of \$10/hour for 10 hours/week during the academic year (360 hours) and 40 hours/week during the summer (560 hours). The students will tweet nice things about Persons One and Two.
\newline
\begin{table}[H]
\begin{tabular}[c]{cccc}
Year 1: \$18,400 & Year 2: \$18,400 & Year 3: \$9,200 & Total: \textbf{\$46,000}
\end{tabular}
\end{table}

\subsection*{C. Fringe Benefits}
Fringe benefits at the University of Whatevs are calculated at a rate of 35\% for faculty and 8\% for students. Following these guidelines, we request the following amounts for benefits.
\newline
\begin{table}[H]
\begin{tabular}[c]{r r r r | r}
                & Year 1     & Year 2     & Year 3     & Total\\
 Person One     & \$3,500    & \$3,570    & \$3,641    & \$10,711\\
 Person Two     & \$3,500    & \$3,570    & \$3,641    & \$10,711\\
 PhD Student    & \$2,000    & \$2,040    & \$2,081    & \$6,121\\
 Ugrad Student  & \$1,472    & \$1,472    & \$736      & \$3,680\\
 \hline
 Total          & \$10,472   & \$10,652   & \$10,099   & \textbf{\$31,223} \\
\end{tabular}
\end{table}

\subsection*{D. Equipment}
\begin{pappg}
    Equipment is defined as tangible personal property (including
    information technology systems) having a useful life of more than one
    year and a per-unit acquisition cost which equals or exceeds the lesser
    of the capitalization level established by the proposer for financial
    statement purposes, or \$5,000. It is important to note that the
    acquisition cost of equipment includes modifications, attachments, and
    accessories necessary to make the property usable for the purpose for
    which it will be purchased. Items of needed equipment must be adequately
    justified, listed individually by description and estimated cost.

    Allowable items ordinarily will be limited to research equipment and
    apparatus not already available for the conduct of the work. General
    purpose equipment such as office equipment and furnishings, and
    information technology equipment and systems are typically not eligible
    for direct cost support. Special purpose or scientific use computers or
    associated hardware and software, however, may be requested as items of
    equipment when necessary to accomplish the project objectives and not
    otherwise reasonably available. Any request to support such items must
    be clearly disclosed in the proposal budget, justified in the budget
    justification, and be included in the NSF award budget. See 2 CFR §
    200.313 for additional information. \textbf{(iv) Travel (Line E on the
    Proposal Budget) }

\end{pappg}
We will purchase a powerful workstation to play on Facebook. We will purchase 200TB of archive storage from the University of Whatevs to enable the dissemination of memes and gifs.
\newline
\begin{table}[H]
\begin{tabular}[c]{r | r}
 & Total\\
Workstation     & \$10,000\\
200 TB Storage  & \$32,000\\
\hline
Total           & \textbf{\$42,000}
\end{tabular}
\end{table}

\subsection*{E. Travel}
PI Person One, PI Person Two, and the PhD student plan to attend vacations in small-town America. We request the following travel funds to cover airfare, per diem, and booze.
\newline
\begin{table}[H]
\begin{tabular}[c]{cccc}
Year 1: \$4,400  & Year 2: \$4,400    & Year 3: \$4,400    & Total: \textbf{\$13,200}
\end{tabular}
\end{table}

\subsection*{E. Participant Support Costs}
There are no participant support costs associated with this project.

\subsection*{G. Other Direct Costs}
We plan to publish the results of our study in Years 2 and 3. We project that results will be verifiable and provide data that will produce information which will be disseminated through conferences where people waste time hearing themselves talk and publication in obscure journals that no one actually reads. A desktop computer will be used to conduct this work and to read TMZ when we are bored. Additionally, we anticipate the need for external local disk storage, software licenses, and repair fees for existing workstations. We also are requesting funds in each project year to support miscellaneous costs associated with the outreach activities, such as Snapchat filters.  These costs may include: basic supplies (notebooks, clipboards, etc.) and subscriptions to HBO Now and Showtime Anytime.
\newline
\begin{table}[H]
\begin{tabular}[c]{r r r r | r}
 & Year 1 & Year 2 & Year 3 & Total\\
Publication Charges     & \$0        & \$3,000    & \$3,000    & \$6,000\\
Desktop Computer        & \$2,750    & \$0        & \$0        & \$2,750\\
Technology Supplies     & \$1,000    & \$1,000    &\$1,000     & \$3,000\\
Outreach Supplies       & \$2,000    & \$2,000    &\$2,000     & \$6,000\\
\hline
Total                   & \$5,750   & \$6,000   & \$6,000 & \textbf{\$17,750}
\end{tabular}
\end{table}

\subsection*{H. Indirect Costs}

\noindent The modified total direct cost for each year was multiplied by the corresponding negotiated overhead rate. The indirect costs are summarized below.
\newline
\newline
\begin{table}[H]
\begin{tabular}[c]{r r r | r}
        & Direct Costs  & Rate      & Total\\
Year 1  & \$84,022      & 51.0\%    & \$42,851\\
Year 2  & \$85,352      & 51.5\%    & \$43,956\\ 
Year 3  & \$76,517      & 52.5\%    & \$40,171\\
\hline
        &               & Total     & \textbf{\$126,978}
\end{tabular}
\end{table}
